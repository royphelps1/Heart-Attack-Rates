\documentclass[11pt]{article}
\usepackage[a4paper,margin=1in]{geometry}
\usepackage[T1]{fontenc}
\usepackage{lmodern}
\usepackage{hyperref}
\usepackage{parskip}
\usepackage{enumitem}
\setlist[itemize]{topsep=2pt,itemsep=2pt,parsep=0pt,partopsep=0pt}

\title{DSCI 511 -- Project Proposal}
\author{Roy Phelps \and Shad Scarboro \and Leland Weeks \and Evan Wessel}
\date{\today}

\begin{document}
\maketitle

\section{Team Formation and Roles}
Our team consists of four members with complementary strengths. Below are our self-identified skills and anticipated contributions:

\textbf{Roy Phelps}\\
\emph{Skills:} Python, Jupyter Notebooks, basic data cleaning, visualization\\
\emph{Contribution:} Code development, documentation, Git/GitHub organization

\medskip
\textbf{Shad Scarboro}\\
\emph{Skills:} Data sourcing, research, presentation formatting\\
\emph{Contribution:} Lead on data acquisition planning and writeup support

\medskip
\textbf{Leland Weeks}\\
\emph{Skills:} Python scripting, exploratory analysis, math/stats foundations\\
\emph{Contribution:} Support on initial dataset exploration and preprocessing

\medskip
\textbf{Evan Wessel}\\
\emph{Skills:} Writing, data summaries, editing\\
\emph{Contribution:} Drafting sections of the proposal and summarizing findings

\section{Data Source of Interest (To Be Finalized)}
We are in the process of selecting a dataset topic. The dataset will be chosen based on the following criteria:
\begin{itemize}
    \item Real-world relevance and potential impact
    \item Feasibility of acquisition within the course timeline
    \item Opportunities for exploratory analysis and coding
    \item Clear connection to potential users or applications
\end{itemize}

\noindent\textbf{Potential areas we are considering:}
\begin{itemize}
    \item Health \& wellness data
    \item Environmental or climate data
    \item Sports statistics
    \item Social media or public sentiment
    \item Education and academic performance
    \item Transportation or mobility data
\end{itemize}

\noindent Once we finalize a topic, we will specify:
\begin{itemize}
    \item The exact source (API, website, files, etc.)
    \item The expected format (CSV, JSON, text, etc.)
    \item The licensing or access constraints, if any
\end{itemize}

\section{Potential Users and Applications}
We intend to collect a dataset that can be useful beyond this class. Once our topic is chosen, we will identify:
\begin{itemize}
    \item Who might use this data (e.g., researchers, policy analysts, educators, businesses, developers)
    \item Example use cases (e.g., modeling, dashboards, comparison studies, predictive insights)
    \item Academic or commercial relevance
\end{itemize}

\section{Plan for Acquiring the Data}
Our plan will depend on the chosen topic, but we anticipate one of the following acquisition methods:
\begin{itemize}
    \item Downloading CSV/JSON files from open data portals or repositories
    \item Using a public API and writing Python scripts to fetch the data
    \item Web scraping (if allowed and ethical)
    \item Combining multiple smaller datasets into a single structured source
\end{itemize}

\noindent After finalizing the dataset, we will document:
\begin{itemize}
    \item Tools we plan to use (e.g., Python, BeautifulSoup, requests, pandas)
    \item Steps to handle cleaning and preprocessing
    \item Any potential obstacles (e.g., limits, formatting issues, permissions)
\end{itemize}

\section{Sample of the Dataset (Placeholder)}
A preliminary data sample will be included once our source is selected. This may include:
\begin{itemize}
    \item A few sample rows
    \item A screenshot, snippet, or small extracted file
    \item A short block of test code if we access via API or scraping
\end{itemize}

\noindent Example placeholder until the dataset is confirmed:
\begin{verbatim}
[Sample data will be inserted here once the source is chosen.]
Possible formats: CSV snippet, JSON object, or scraped output
\end{verbatim}

\section{Submission Plan}
We will submit the proposal as a Markdown or PDF document (or via Jupyter Notebook if preferred) including:
\begin{itemize}
    \item All required sections
    \item Team member names and roles
    \item Dataset description and acquisition plan
    \item Sample or preview of the dataset
\end{itemize}

\noindent Once our topic is locked in, we will revise this template and finalize the language.

\end{document}
